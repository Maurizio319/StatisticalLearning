\chapter{Simulating data from a GLM in R}
Simulating data is often used to test the performance of your model.

\section{Example: simulating Poisson data}
In general, in a Possion regression you have:
$$Y_i = (Y|\vec{X} = \vec{x}_i) \sim \text{Poisson}(\mu_i)$$
$$\mu = E[Y_i] \in (0, +\infty)$$
$$\eta_i = \vec{x}_i^t\vec{\beta} \in (-\infty, \infty)$$

If you then decide to use the canonical link:
$$\eta_i = \log(\mu_i) \implies \mu_i = e^{\eta_i}$$

Therefore you can generalize this Poisson regression as
$$Y_i \sim \text{Poisson}(e^{\vec{x}_i^t\vec{\beta}})$$

To then simulate data to test this model you must:
\begin{itemize}
  \item Assume a number of predictors $p$(in our case $p=1$ for plotting convenience)
  \item Fix some values for $\beta$ (in our case notice that $\eta_i = \beta_0 + \beta_ix_i$)
  \item Generate $n$ random values of $x_i$ (in this case $n=100$) and compute the corresponding $y_i$ values.
\end{itemize}

In R this translates to:
\begin{minted}{R}
# Simulate the data
set.seed(123)  # In order to get repeatable results
x <- rnorm(100, 0, 2)  # Does not matter, just to get random values 
beta0 <- 1  # Fixed
beta1 <- 2  # Fixed
eta <- beta0 + beta1 * x  # Vector of values
mu <- exp(eta)  # Because we chose the canonical link
y <- rpois(100, lambda=mu)  # Y values corresponding to random X values

# Ignoring the model, try to predict relation
plot(x, y)  # See that is not linear
plot(x, log(y))  # Since we obtain a line, we expect eta = beta0 + beta1x

# Try fitting linear model (not the best approach)
z <- log(y)  # Not good since you have many zeros and infinites
summary(z)  
z <- log(y+1)  # Sudo-count (commonly used)
summary(z)
modA <- lm(formula=z~x)  # Fit model
summary(modA)  # Estimated parameters are far from chosen ones
plot(modA)  # Notice systematic trends in residual analysis 
modB <- lm(z~x+I(x^2))  # Try adding quadratic term to match residual analysis
# Note: I(math_exp) is used to interpret the content as a mathematic expression
summary(modB)  # Not that far from starting data
plot(modB)  # Not the best but seems decent

# Proper way to try and fit data
y[1:100]  # Inspect the data and notice that the response in positive discrete
# You could then try fitting different distributions based on that, ...
# ... Poisson seems reasonable
modC <- glm(formula=y~x,family="poisson")  # Canonical link by default
summary(modC)  # Estimates are rather close
# Note: Because of how glm works, you need decently big values for n
# Note: Fisher Scoring gives info on optimization attempts number
plot(modC)  # Seems decent
# Note: Some regular-ish patterns may be seen when working with discrete data

# Use model to make predictions
p1<-predict(modC)  # ! Estimates linear predictors, not values of y
summary(p1)  
p2<-predict(modC,type="response")  # Estimates values of y
summary(p2)
plot(p2,y)  # Plot predicted vs simulated values of y
abline(0,1)  # Line that should be approximated by the datapoints 
\end{minted}

\section{Example: simulating exponential data}
You can do the same with other distributions. This is the example for an exponential distribution.
\begin{minted}{R}
x <- rnorm(100, 0, 2)  # Does not matter, just to get random values 
beta0 <- 1  # Fixed
beta1 <- 2  # Fixed
eta <- beta0 + beta1 * x  # Vector of values
mu <- exp(eta) / (1 + exp(eta))  # Use logit as link
y <- rbinom(100, size=1, prob=mu)  # Compute corresponding y values
plot(x, y)

modD <- glm(y~x, family="binomial")  # Create the model
# modD <- glm(y~x, family=binomial(link="logit"))  # Equivalent formulation

p4<-predict(modD) # Compute linear predictors
summary(p4)
p5<-predict(modD,type="response") # Compute fitted probabilities
summary(p5)
\end{minted}
