\chapter{Generalized linear models}
  \textbf{Generalized linear models} (GLMs) are a broad family of regression models which includes, for instance, linear, logistic and poisson regressions (and many more). The concept was created by Nelder and Wedderburn in 1972, in order to map into a single framework all regressions models.
  In general regression models are characterized by:
  \begin{itemize}
    \item $Y$: the random \textbf{response variable} you want to predict
    \item $\vec{X}=(X_1, \dots, X_p)$: a \textbf{vector of random predictors} (generally known)
  \end{itemize}
  Notice that, since these models describe the value of $Y$ with respect to the vector $\vec{X}$ ($Y|\vec{X}=\vec{x}$), they do not give information on the actual distribution of $Y$, but rather on the values that it takes conditionally to the values taken by the predictors, which can be written as
  $$Y_i=(Y|\vec{X}=\vec{x}_i)$$
  Therefore, if you have $n$ vectors of random predictors, $Y_1, \dots, Y_n$ are independent random variables but not identically distributed.
  %TODO Add images

  \section{From linear models to generalized linear models}
    Linear models make some implicit assumptions that must be removed in order to generalize.
    
    \begin{tabularx}{1\textwidth} { 
    >{\centering\arraybackslash}X 
    >{\centering\arraybackslash}X }

    \textbf{Linear models} & 
    \textbf{Generalized linear models} \\

    $Y_i \sim N(\mu_i, \sigma^2)$ & 
    $Y_i \sim f(y_i; \mu_i, \phi)$ \\

    $\mu_i = E[Y_i] = E[Y|\vec{X} = \vec{x}_i] = \eta_i$ & 
    $g(\mu_i) = \eta_i$ \\

    $\eta_i = \vec{x}_i^t\vec{\beta} = \beta_0x_{i0} + \beta_1x_{i1} + \dots + \beta_px_{ip}$ & 
    $\eta_i = \vec{x}_i^t\vec{\beta}$  \\
    \end{tabularx}

    First of all, linear models assume that the random variable $Y_i$ is normal distributed, with mean $\mu_i$, meaning that $E[Y|\vec{X}=\vec{x}_i]$, and $\sigma^2 = Var(Y_i) = Var(Y|\vec{X}=\vec{x}_i)$. Notice that the variance is constant, therefore homoskedasticity is assumed. On the other hand, generalized linear models do not make any assumptions on the shape of the distribution of $Y_i$; $Y_i$ is distributed according to some function $f(y_i)$ with the parameters $\mu_i$ and $\phi$. $\phi$ is called \textbf{special parameter} and it may or may not be present in the model.
    The \textbf{linear predictor} is the function used to predict the outcome of the random variable $Y_i$ knowing some values for the predictor $\vec{X}_i$; in both cases the linear predictor is defined as $\eta_i = \vec{x}_i^t\vec{\beta} = \beta_0x_{i0} + \beta_1x_{i1} + \dots + \beta_px_{ip}$, where $p$ is the number of parameters and the coefficients $\beta_0, \dots, \beta_p$ are the weights for the predictors (therefore the linear predictor is a linear combination in $\vec{\beta}$ of the predictors). The difference is in the relation between $\mu_i$ and $\eta_i$. In linear models $\mu_i$ coincides with $\eta_i$, because assuming a normal distribution $\mu_i = E[Y_i] = E[Y|\vec{X} = \vec{x}_i] = \eta_i$; in generalized linear models $\mu_i$ and $\eta_i$ are related by a \textbf{link function}, which is a function $g$ of $\mu_i$ such that $g(\mu_i) = \eta_i$.
    A generalized linear model is then a choice of a distribution function $f$ and a link function $g$; for instance, if $f$ is a normal distribution (with $\sigma^2 = \phi$) and $g$ is the identity function (therefore $\mu_i = \eta_i$), we get the linear model. 
  
  \section{Exponential dispersion family}
    The function $f$, describing the distribution of the random variable $Y_i$, needs to be a member of the \textbf{exponential dispersion family} (EDF). A density belonging to the EDF can be written as
    $$f(y_i;\theta_i,\phi)=\exp\left\{\frac{y_i\theta_i-b(\theta_i)}{a_i(\phi)}+c(y_i;\phi)\right\}$$
    With:
    \begin{itemize}
      \item $f(y_i)$ is a generic density function 
      \item $\theta_i$ and $\phi$ are some parameters. The values of $\theta_i$ are called canonical parameters.
      \item $a_i$, $b$ and $c$ are some functions.
    \end{itemize}

    \subsection{Example: normal distribution is a member of the EDF}
      We want to demonstrate that the normal distribution is a member of the EDF.
      \begin{align*}
      f(y_i; \mu_i, \sigma^2)
        &= \frac{1}{\sqrt{2\pi\sigma^2}} \exp\left\{-\frac{1}{2} \frac{(y_i-\mu_i)^2}{\sigma^2}\right\}
        & \text{ definition of normal distribution}\\
        &= \exp\left\{-\frac{1}{2} \frac{(y_i-\mu_i)^2}{\sigma^2} + \log\left(\frac{1}{\sqrt{2\pi\sigma^2}}\right)\right\}
        & \text{ pull coefficient inside the exponential}\\
        &= \exp\left\{-\frac{1}{2} \frac{(y_i-\mu_i)^2}{\sigma^2} -\frac{1}{2}\log\left(2\pi\sigma^2\right)\right\}
        & \text{ rearrange using log properties}\\
        &= \exp\left\{-\frac{1}{2} \frac{y_i^2-2y_i\mu_i+\mu_i^2}{\sigma^2} -\frac{1}{2}\log\left(2\pi\sigma^2\right)\right\}
        & \text{ compute the square}\\
        &= \exp\left\{\frac{y_i\mu_i+\nicefrac{\mu_i^2}{2}}{\sigma^2} - \frac{1}{2\sigma^2}y_i^2 - \frac{1}{2}\log\left(2\pi\sigma^2\right)\right\}
        & \text{ group terms containing } \mu_i\\
      \end{align*}
      The normal distribution therefore belongs to the EDF by setting:
      \begin{align*}
        \theta_i     &= \mu_i, \\
        \phi         &= \sigma^2, \\
        b(\theta_i)  &= \nicefrac{\theta_i^2}{2}, \\
        a_i(\phi)    &= \phi, \\
        c(y_i; \phi) &= - \frac{1}{2\phi} y_i^2 - \frac{1}{2}\log(2\pi\phi)
      \end{align*}

    \subsection{Example: binomial distribution is a member of the EDF}
      We want to demonstrate that the binomial distribution is a member of the EDF.
      \begin{align*}
      f(y_i; n_i, \pi_i)
        &= \binom{\pi_i}{y_i}\pi_i^{y_i}(1-\pi_i)^{n_i-y_i}
        & \text{ definition of binomial distribution}\\
        &= \exp\,\left\{\log\,\left(\binom{\pi_i}{y_i}\pi_i^{y_i}(1-\pi_i)^{n_i-y_i}\right)\right\}
        & \text{ exponentiate to compare with EDF} \\
        &= \exp\,\left\{y_i\log\pi_i+(n_i-y_i)\log(1-\pi_i)+\log\binom{n_i}{y_i}\right\}
        & \text{ split log and simplify}\\
        &= \exp\,\left\{\frac{y_i\log\left(\frac{\pi_i}{1-\pi_i}\right)+n_i\log(1-\pi_i)}{1}+\log\binom{n_i}{y_i}\right\}
        & \text{ group terms containing } \pi_i\\
      \end{align*}
      
      The binomial distribution therefore belongs to the EDF by setting:
      \begin{align*}
        \theta_i     &= \log\left(\frac{\pi}{1-\pi}\right) \text{(logit)} \\
        \phi         &= 1 \\
        b(\theta_i)  &= - n_i \log(1-\pi_i) \\
                     &= -n_i \log(1-\frac{e^{\theta_i}}{1 + e^{\theta_i}}) \\
                     &= n_i \log(1 + e^{\theta_i}) \\
        a_i(\phi)    &= \phi \\
        c(y_i; \phi) &= \log\binom{n_i}{y_i}
      \end{align*}

      For the binomial distribution ($Y_i \sim Bin(n_i, \pi_i)$) it is possble to write the relation between old and new parameters explicitely
      Link from $\theta_i$ to $\mu_i$.
      $$E[Y_i] = \mu_i = n_i\pi_i = n_i\frac{e^{\theta_i}}{1 + e^{\theta_i}}$$
      Link from $\mu_i$ to $\theta_i$.
      $$\pi_i = \frac{\mu_i}{n_i} \implies \theta_i = \log\left(\frac{\pi_i}{1-\pi_i}\right) = \log\left(\frac{\nicefrac{\mu_i}{n_i}}{1-\nicefrac{\mu_i}{n_i}}\right)=\log\left(\frac{\mu_i}{n_i-\mu_i}\right)$$
      Notice that \textbf{heteroskedasticity} in naturally embedded in the model since the variance depens on $\mu_i$:
      $$Var(Y_i) = n_i\pi_i(1-\pi_i) = n_i\frac{\mu_i}{n_i}\left(1-\frac{\mu_i}{n_i}\right) = \mu_i\left(\frac{n_i - \mu_i}{n_i}\right)$$
      This hold true for all GLMs but linear models which is the only homoskedastic model of the group. The fact that heteroskedasticity is mandatory is not a big problem since most datasets are not homoskedastic anyway.

  \section{Connection between $\theta_i$ and ($\mu_i$, $Var(Y_i)$)}
    In general, one can show that, for any function in the EDF,
    $$\mu_i = E[Y_i] = b^\prime(\theta_i)$$
    Where $b^\prime$ is the first derivative of the function $b$.
    \begin{align*}
    Var(Y_i) &= b^{\prime\prime}(\theta_i) \cdot a_i(\phi) \\
             &= V(\mu_i) \cdot a_i(\phi) \\
    \end{align*}
    Where $b^{\prime\prime}$ is the second derivative of $b$ and $V(\mu_i)$ is the variance function (basically just a way to rewrite underlining the dependence on $\mu$).

    \subsection{Example: connection of $\theta_i$, $\mu_i$ and $Var(Y_i)$ for the binomial model}
      We know, for the binomial function, that
      $$b(\theta_i) = n_i\log(1+e^{\theta_i})$$
      The first derivative in $\theta_i$ then becomes
      $$b^\prime(\theta_i) = \frac{n_ie^{\theta_i}}{1+e^{\theta_i}} = n_i\pi_i$$
      But it is also known that
      $$\mu_i = n_i\pi_i$$
      Therefore, as expected from the general formula
      $$\mu_i = b^\prime(\theta_i)$$
      
      We know that, for the binomial model
      $$a_i(\phi) = 1$$
      We then compute the second derivative of $b(\theta_i)$ in $\theta_i$:
      \begin{align*}
      b^{\prime\prime}(\theta_i) 
        &= n_ie^{\theta_i}\left[-\left(\frac{1}{1+e^{\theta_i}}\right)^2e^{\theta_i}\right] + n_i\frac{e^{\theta_i}}{1+e^{\theta_i}}
        & \text{ by deriving } b^\prime(\theta_i)\\
        &= -n_i\left(\frac{e^{\theta_i}}{1+e^{\theta_i}}\right)^2 + n_i\frac{e^{\theta_i}}{1+e^{\theta_i}}
        & \text{ pulling both } e^{\theta_i} \text{ inside the square}\\
        &= -n_i\pi_i^2+n_i\pi_i
        & \text{ since } \frac{e^{\theta_i}}{1+e^{\theta_i}} = \pi_i\\
        &= n_i\pi_i(1-\pi_i)
        & \text{ collecting } n_i\pi_i\\
      \end{align*}
      Therefore, as expected, we get
      $$Var(Y_i) = b^{\prime\prime}(\theta_i)a_i(\phi) = n_i\pi_i(1-\pi_i)$$

  \section{Choice of link function}
    The choice of link function is not unique and can depend on the function $f$ which was chosen.

    \subsection{Example: link function for a bernoulli function}
      Assume that $Y_i$ is bernoulli distributed with probability $\pi_i$ ($Y_i \sim Ber(\pi_i)$). Then:
      \begin{itemize}
        \item $\mu_i = \pi_i \in (0,1)$ (since $n_i = 1$)
        \item $Var(Y_i) = \pi_i(1-\pi_i) = \mu_i(1-\mu_i)$
        \item $\eta_i = \vec{x}_i^t\vec{\beta}$ with $\eta_i \in (-\infty, +\infty)$
      \end{itemize}
      We must find a function $g$ such that $g(\mu_i)=\eta_i$ which, taking into account that $\mu_i \in (0, 1)$ and $\eta_i \in (-\infty, +\infty)$, means:
      \begin{align*}
        g&: (0, 1) \to (-\infty, +\infty) \\
        g^{-1}&: (-\infty, +\infty) \to (0, 1)  \\
      \end{align*}
      These conditions are satisfied by any cumulative density functions (CDF), as long as it is continuous (since it must be invertible).
      We therefore have infinitely many functions $g$ that satify these conditions, but the three most common choices are:
      \begin{enumerate}
        \item \textbf{Standard normal CDF}: Choose $g^{-1}=\Phi$, with $\Phi \sim N(1,0)$, hence
          $$g = \Phi^{-1} : (0,1) \to (-\infty, +\infty)$$
          This type of regression ($f \sim \text{ bernoulli }, g = \Phi^{-1}$) is called \textbf{probit regression}.
        \item \textbf{Logistic CDF}: Consider the logistic distribution, which has density
          $$f(y; \mu, \sigma^2) = \frac{\exp(\nicefrac{(y-\mu)}{\sigma})}{\sigma^2(1 + \exp\left(\nicefrac{(y-\mu)}{\sigma}\right)^2}\;,\; \text{ with } -\infty < y < +\infty$$
          Setting $\mu = 0$, $\sigma^2 = 1$ we get the sigmoid function
          $$g^{-1} = F(y; 0, 1) = \frac{e^y}{1 + e^y}$$
          The link function then becomes the logit link
          $$\frac{e^\eta}{1 + e^\eta} = \mu \implies \eta = \log\left(\frac{\mu}{1-\mu}\right)$$
          This regression ($f \sim \text{ bernoulli }, g = \text{ logit }$) is the \textbf{logistic regression}. 
        \item \textbf{Poisson CDF}: This link function is rarely used since it performs well in niche cases, such as with a binomial distributions with $\pi \sim \text{ Bernoulli }$ (e.g. the number of bacteria on a plate in a dilution bioassay, in which you do not measure the number of bacteria, just whether there are or not). Assume that $Z_i$ is a possion distributed random variable with parameter $\lambda_i$ ($Z_i \sim \text{ Poisson }(\lambda_i)$). Then assume that we measure
        $$
        y_i = 
        \begin{cases}
          0 & \text{ if } z_i = 0 \text{ (absent)} \\
          1 & \text{ if }  z_i > 0 \text{ (present)}
        \end{cases}
        $$
        We would like to describe how $y_i$ changes with time ($x = 0, 1, 2, \dots$), therefore we need a function that links $\mu_i$ and $\eta_i$.
        Since we know that 
        $$\eta_i = \log(\lambda_i)$$
        and that
        \begin{align*}
        \mu_i 
          &= E[Y_i] \\
          &= 0 \cdot P(Y_i = 0) + 1 \cdot P(Y_i = 1) \\
          &= P(z_i > 0) \\ 
          &= 1 - P(z_i = 0) \\
          &= 1 - e^{-\lambda_i}
        \end{align*}
        we can link $\mu_i$ and $\eta_i$ passing through $\lambda_i$ ($\mu_i \leftrightarrow \lambda_i \leftrightarrow \eta_i$). Firstly we rearrange the link between $\mu_i$ and $\lambda_i$:
        $$\mu_i = 1 - e^{-\lambda_i} \implies e^{-\lambda_i} = 1 - \mu_i \implies \lambda_i = -\log(1 - \mu_i)$$
        Then, plugging in the link between $\lambda_i$ and $\eta_i$ ($\log(\lambda_i) = \eta_i$) we get
        $$\eta_i = \log(-\log(1-\mu_i))$$
        This link is called \textbf{complementary log-log link}
      \end{enumerate}

    \subsection{Canonical link}
      The logit link is also called \textbf{canonical link} since 
      $$\eta_i = \log\left(\frac{\mu_i}{1-\mu_i}\right) = \theta_i$$
      In general a canonical link is a function g such that 
      $$\eta_i = g(\mu_i) = \theta_i$$
      These link functions have some advantages:
      \begin{itemize}
        \item Clear interpretation of the parameters ($\vec{\beta}$)
        \item Some theoretical reasons (Fisher scoring and iteratively reweighted least squares algorithms are equivalent, zero sum residuals)
        \item It follows directly from the EDF formulation of the distribution (just derive), in fact $\mu_i = b^\prime(\theta_i), \theta_i = (b^\prime)^{-1}(\mu_i)$
      \end{itemize}
      Despite these advantages the logit link is not always the correct one to use.
